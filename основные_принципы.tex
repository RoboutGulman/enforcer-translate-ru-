\begin{multicols}{2}
    [
    \section{Основные принципы}
    ]
    \textbf{Броски кубов}

    Все проверки в игре делаются с помощью броска куба д20. Бросок всегда происходит против определённой сложности (число от 1 до 20). 
    Для успешного прохождения проверки нужно выбросить число, меньше или равно сложности.

    Вы можете получать модификаторы к броску (именно к броску, а не характеристике). Например, если вы выкинули 17 на кубе и у вас модификатор -1 к броску, 
    итоговый результат - 16.

    Вы также можете получить благословения и проклятия к броску. За каждый из этих эффектов вы кидаете дополнительный куб д6. Благословения вычитаются из броска, проклятия - прибавляются к броску. 
    Благословения и проклятия взаимоуничтожаются перед броском (например, если у вас 3 благословения и 1 одно проклятие, вы должны будете кидать только 2 благословения).

    Округляйте всегда вниз.

    \textbf{Хоумрул}

    Если вы получили больше одного благословения/ проклятия на бросок, примените только наибольший результат д6. Например, если вы выкинули 10 на д20, 2 и 5 на благословениях, итоговый результат = 10 - 5 = 5.
    \end{multicols}