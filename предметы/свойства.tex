\section*{Свойства оружия}
    \begin{multicols}{2}
        \textbf{1.5H}

        Это оружие может использоваться одной или двумя руками. Если оно используется одной рукой, то обладатель получает проклятие на BS/WS

        \textbf{2H}

        Оружие может использоваться только двумя руками

        \textbf{AOE}

        После того, как вы попали стрелковой атакой с этим свойством в цель, вы должны поставить шаблон соответствующего диаметра центром над целью.
        Все бойцы под этим шаблоном получают соответствующий урон. При промахе происходит смещение шаблона, а затем все бойцы под ним также получают урон. 

        \textbf{Bleeding}

        Может наложить кровотечение на цель.

        \textbf{Burning}

        Может наложить горение на цель.

        \textbf{Burst shot}

        Может потратить 3 боеприпаса, но получает благословение на BS.

        \textbf{Cumbersome}

        2H + оружие не может стрелять, если персонаж двигался в этом ходу.

        \textbf{Disoriented}

        Может наложит статус Дезориентирован

        \textbf{EMP}

        Это свойство поглощает все боеприпасы из las и plasma оружия. Все механ. миньоны, а также персонажи в Exosuit и с Аугметикой получают статус Оглушён.

        \textbf{Full Auto}

        Может потратить 10 патронов и получить 2 благословения на BS.

        \textbf{Launcher}

        Может использоваться только с Missle Launcher

        \textbf{Obscure}

        Оставляет облако дыма, которое даёт проклятия на BS проверки. (в конце первого хода облако остаётся на 2+, далее на 6+)

        \textbf{Overload}

        Данное оружие может использовать в 2 раза больше боеприпасов (это можно кобминировать с Burst и Full auto), и при этом наносит удвоенный урон.

        \textbf{Parry}

        Оппонент в ближнем бою получает дополнительный штраф +1 к WS.

        \textbf{Unstable}

        Если на попадание этим оружием выпадает 1 или 20 (без учёта модификаторов), вы должны потратить половину от максимума боеприпасов этого оружия.
        Если вы не хотите (или не можете) тратить часть боеприпасов, вы должны получить эквивалентное количество урона.
        Например, Plasma Gun при перегреве должен потратить 10 боеприпасов или его носитель должен получить 10 урона в любой комбинации.

        \textbf{Poisoned}

        Может наложить Отравление

        \textbf{Punching}

        Если вы имеете оружие с таким свойством, вы можете заявлять действия, требующие свободной руки, даже если ваши руки заняты.

        \textbf{Scatter}

        Промах этим оружием приводит к смещению

        \textbf{Stun}

        Может наложить оглушение

        \textbf{Template}

        Данное оружие не требует проверки BS для стрельбы. Все цели под шаблоном получают автоматическое попадание.

        \textbf{Trap}

        Это оружие не взрывается сразу же при выбрасывании. Оно взрывается когда противник подходит в AOE дальность.

        \textbf{Melta}

        Когда цель этого оружия находится на Ближней дистанции этого оружия, при попадании оружие автоматически совершает критическое попадание.

    \end{multicols}