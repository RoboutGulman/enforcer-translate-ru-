\begin{multicols}{2}
    [
    \section{Ходы}
    ]
    \textbf{Порядок хода}

    Система использует Попкорн-инициативу. После начала сценария, игроки выбирают, кто из них ходит первый 
    (исключением являются моменты, когда ДМ приготовил специальный сценарий по типу Засады). Когда выбранный 
    игрок заканчивает ход, он выбирает, кто из игровых персонажей или НПЦ ходит следующим. Это продолжается 
    по тому же принципу: игрок или НПЦ по завершению своего хода выбирает, кто ходит следующим. Когда последняя
     модель на столе завершила свою активацию, раунд заканчивается. Персонаж игрока, который ходил последним
      в этом раунде, выбирает, кто будет ходить первым в следующем (он может выбрать в том числе и себя).

    Когда вы выбираете следующую модель для активации, главное правило – нельзя выбирать модель, которая уже 
    поактивировалась в этом раунде.

    \textbf{Действия}

    Каждую свою активацию, персонаж может совершить 2 действия. Персонаж может пропустить 1 или 2 действия за 
    активацию, но не может сделать больше 2 действий за ход (не считая перков).

    \textbf{Спрятать/достать оружие.}

    Технически это не совсем действие, но это реализация одной из игровых механик. Один раз за ход персонаж может 
    спрятать или достать оружие без траты действий. Также персонаж может сменить оружие без траты действий. Вы можете
    использовать оружие только если оно не спрятано.

    \textbf{Сражаться}
    см. ближний бой

    \textbf{Стрелять}
    см. стрельба

    \textbf{Движение}

    Переместите модель на расстояние, которое равно или меньше Ловкости бойца. Труднопроходимый террейн ополовинивает движение.

    \textbf{Прыжки}

    Персонаж может прыгать горизонтально на дистанцию, равную половине его Ловкости и вертикально на дистанцию, равную четверти 
    его Ловкости. Прыжок может быть выполнен в ходе обычного движения при условии, что суммарная дистанция движения не превышает
    дистанцию прыжка.

    \textbf{Слушать}

    Потратить действие, чтобы пройти проверку Внимательности и при успехе раскрыть всех вражеских бойцов в пределах дальности, 
    равной хар-ке Внимательности.

    \textbf{Прицеливание}

    Потратить действие, чтобы прицелиться в определённый объект (можно прицеливаться в зону не больше 5СМ в диаметре). Получить 
    благословение к следующей проверке стрельбы против цели. Бонусы от прицеливания суммируются, но как только боец подвигался,
    они пропадают. Если боец атакован стрелковой атакой, он должен пройти тест воли(в противном случае боец теряет прицеливание).

    \textbf{Перезарядка}

    Если у вас есть сменный магазин, вы можете потратить действие и перезарядить своё оружие.

    \textbf{Предупреждающий огонь.}

    Персонаж незамедлительно заканчивает свою активацию, но стреляет по первому противнику в зоне видимости, который ходит, стреляет 
    или дерётся в ближнем бою. Это считается обычной стрелковой атакой.

    \textbf{Подавляющий огонь}

    Выберите в качестве цели объект (можно прицеливаться в зону не больше 5СМ в диаметре) для подавления. Любой боец, являющийся целью 
    или любой боец между стрелком и целью получает статус «Залёг». Подавляющий огонь не наносит урона и тратит половину боеприпасов 
    оружия (это значит, что для заявления этого действия боец должен держать оружие, у которого осталось хотя бы половина боеприпасов).

    \textbf{Залечь в укрытие}

    Персонаж заканчивает свой ход и занимает защищённую позицию. Враги, атакующие персонажа, получают Проклятие на проверки стрельбы 
    и персонаж получает иммунитет к критам.

    \textbf{Использовать пси-силу}
    см. пси-силы

    \textbf{Медицинская помощь}

    Оказать медицинскую помощь себе или другому персонажу. Требуется мед.кит и проверка HL. Варианты использования:
    \begin{itemize}
        \item Чтобы восполнить HP: Удачная проверка HL восстанавливает 6 HP, неудачная – 2
        \item Чтобы восстановить тяжелораненную цель: Успех восстанавливает цель и поднимает её HP до 1. 
        Провал – цель остаётся тяжелораненной, но восстанавливает 2 HP, до максимума в 0 HP
        \item Чтобы стабилизировать умирающую цель: Успех стабилизирует цель, даёт ей статус тяжелоранена и поднимает её HP до 0. Провал не оказывает эффекта.
    \end{itemize}
    При любом из выбранных вариантов вы также можете снять любой эффект с бойца (вне зависимости от результата проверки).

    \textbf{Молитва}

    Если у персонажа есть перк Prayer, он может заявить действие Молитва. Данное действие можно совершать только 1 раз за раунд. Каждое действие направлено только на одного бойца.
\end{multicols}

\begin{table}[H]
    \centering
    \begin{tabular}{|l|l|l|l|}
    \hline
    Молитва                                                     & Длительность & Дальность                                                 & Описание                                                                                                                                            \\ \hline
    Помазание                                                   & Мгновенная   & Касание                                                   & \begin{tabular}[c]{@{}l@{}}Цель может пройти проверку HL.  \\ При успехе, она может излечить  \\ 1 HP или избавиться от одного статуса\end{tabular} \\ \hline
    Благословение                                               & 1 Раунд      & \begin{tabular}[c]{@{}l@{}}Линия\\ видимости\end{tabular} & Увеличивает все шансы на крит цели на 2                                                                                                             \\ \hline
    Заступничество                                              & 2 Раунда     & \begin{tabular}[c]{@{}l@{}}Линия\\ видимости\end{tabular} & Цель может перебрасывать 1 проверку за раунд                                                                                                        \\ \hline
    \begin{tabular}[c]{@{}l@{}}Последние\\ почести\end{tabular} & Мгновенная   & Касание                                                   & \begin{tabular}[c]{@{}l@{}}Душа цели отправляется к Императору. \\ Применимо к умирающим.\end{tabular}                                              \\ \hline
    \end{tabular}
    \end{table}