\begin{multicols}{2}
[
\section{Персонажи}
]
\textbf{Создание персонажа}

Вместо классов в игре присутствует ряд архетипов, основанных на типах агентов, которых можно найти в команде 
энфорсеров. Чтобы создать персонажа, выберите архетип и или накиньте характеристики для вашего персонажа, 
или просто запишите характеристики по умолчанию для этого архетипа.

Каждый архетип также идёт с набором экипировки по умолчанию и одним или двумя перками. 
Вы также можете или взять те, что есть, или заменить на свои. Около заголовка экипировки 
и перков вы увидите число. Для экипировки это число очков реквизиции, которые персонаж может 
потратить на получение экипировки. Для перков это количество перков, которое вам доступно. 
Поэтому при кастомном создании персонажа учитывайте только сколько перков может иметь архетип, 
а также сколько реквизиции он может потратить на экипировку.

\textbf{Первичные характеристики}

Каждая из характеристик - число от 1 до 20

\begin{enumerate}
    \item Здоровье (HL)
    \item Ближний бой (WS)
    \item Стрельба (BS)
    \item Ловкость (DX)
    \item Стойкость (TN)
    \item Воля (WL)
    \item Интеллект (IT)
    \item Внимательность (PR)
    \item Лидерство (LD)
\end{enumerate}

\textbf{Вторичные характеристики}

\textbf{Шанс крита}
Каждая первичная характеристика имеет шанс крита. Изначально, он равен 2, но по ходу 
развития персонажа шанс крита может расти. Эта характеристика влияет, насколько сложно 
вашему персонажу получить критический успех при броске. Например, если ваш критический 
шанс для Стрельбы равен 4, вы получите критический успех при выпадении на кубе 1,2,3 и 4.

\textbf{Скорость передвижения}
Ловкость = на сколько ваш персонаж ходит (в сантиметрах)

\textbf{Здоровье (HP) и броня (AP).}
Максимальное количество здоровья равно хар-ке HL. Текущее здоровье уменьшается, когда персонаж получает 
урон (оно может падать ниже 0), и увеличивается, когда персонаж лечится (оно не может стать выше хар-ки HL) 
Броня действует аналогично здоровью (броню можно чинить прямо во время боя), но максимальное кол-во брони 
зависит от типа брони персонажа. Сначала входящий в персонажа урон идёт по броне, и только если брони не осталось, 
начинает уменьшаться здоровье персонажа.

\textbf{Грузоподъёмность}
- сколько вещей ваш персонаж может нести. Равна половине от Стойкости.

\textbf{Очки реквизиции}
Все архетипы имеют трейт REQ. Это цена в очках реквизиции, которые надо потратить, чтобы нанять его.
 Изначально каждый архетип имеет цену в 20. Вы можете поднять значение каждой характеристики архетипа 
 на 1, но это поднимет цену на 10. Например если вы хотите нанять персонажа с +3 ко всем характеристикам,
  вам надо будет заплатить 50 REQ.

Однако, возле заголовка с Экипировкой каждого архетипа вы увидите число. Это число REQ, которые они 
сами могут потратить на закупку экипировки. Так что если вы хотите персонажа, который придёт со своей
 экипировкой, вы должны будете заплатить половину от его REQ стоимости (округляя вверх). Тогда вы 
 получите персонажа и все его REQ . По сути, вы покупаете экипировку персонажа за полцены.

Например, если вы хотите нанять Apprentice энфорсера со всей его экипировкой, а также поднять его 
характеристики на 1, вы должны будете заплатить 50 REQ.

\end{multicols}