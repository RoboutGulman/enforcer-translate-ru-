\section{Бой}
\textbf{Стрельба}

При действии стрельбы вам нужно бросить д20. Если результат выше вашего показателя стрельбы, действие провалено, 
в противном случае оно успешно.

При успехе цель получает урон, равный урону вашего оружия.

Модификаторы стрельбы:

\begin{table}[H]
    \centering
    \begin{tabular}{|l|l|l|}
    \hline
    Описание                                                                                     & Благословение & Проклятие \\ \hline
    Цель в частичном укрытии                                                                     &               & 1         \\ \hline
    Цель в полном укрытии                                                                        &               & 2         \\ \hline
    Цель на дальней дистанции                                                                    &               & 1         \\ \hline
    Цель на средней дистанции                                                                    & -             & -         \\ \hline
    Цель на ближней дистанции                                                                    & 1             &           \\ \hline
    Прицеливание                                                                                 & 1             &           \\ \hline
    Цель двигалась дважды в предыдущем ходу                                                      &               & 1         \\ \hline
    \begin{tabular}[c]{@{}l@{}}Стрелок стоит выше чем цель\\ (минимум 5 СМ разницы)\end{tabular} & 1             &           \\ \hline
    Очередь выстрелов                                                                            & 1             &           \\ \hline
    Автоматический огонь                                                                         & 2             &           \\ \hline
    Атака вне арки видимости цели                                                                & 1             &           \\ \hline
    \end{tabular}
    \end{table}

    \begin{multicols}{2}
        \textbf{Атака вне арки видимости цели}

        Вы получаете благословение, если атаковали извне арки видимости цели. Арка видимости цели - 180 градусов перед моделью.

        \textbf{Шальные выстрелы}

        Любой персонаж, который находится в пределах 2 СМ от линией между стрелком и целью, имеет шанс получить шальной выстрел. 
        Если стрелок проваливает проверку стрельбы и кидает 18-20 , выстрел попадает в ближайшую от стрелка цель, которая 
        находилась на линии огня.

        \textbf{Метательное оружие}

        Метательное оружие (гранаты и ножи) может быть брошено персонажем на дистанцию, равную 1.5 * Стойкость. Вы также как 
        и при обычной стрельбе кидаете проверку стрельбы. Но при провале вы кидаете на смещение. Бросьте 2д6. При дубле, 
        атака попала в цель, как будто вы успешно прошли проверку стрельбы. Иначе, проведите невидимую линию между брошенными 
        кубами, от куба с меньшим значением до куба с большим значением. Дистанция смещения равна удвоенному большему значению.
    \end{multicols}

    \newpage

    \textbf{Ближний бой}

    Ближний бой похож на дальний, но используемая характеристика - не стрельба, а ближний бой.

    Модификаторы ближнего боя:

    \begin{table}[H]
        \centering
        \begin{tabular}{|l|l|l|}
        \hline
        Описание                                                                                                                     & Благословение & Проклятие \\ \hline
        Атакующий находится на возвышенности                                                                                         & 1             &           \\ \hline
        Защищающийся оглушён                                                                                                         & 1             &           \\ \hline
        Атака вне арки видимости цели                                                                                                & 1             &           \\ \hline
        Нападение                                                                                                                    & 1             &           \\ \hline
        \begin{tabular}[c]{@{}l@{}}За каждую единицу Дальности, на которую\\  ваше оружие превосходит оружие противника\end{tabular} & 1             &           \\ \hline
        \begin{tabular}[c]{@{}l@{}}За каждую единицу Дальности, на которую\\ оружие оппонента превосходит ваше\end{tabular}          & 1             & 1         \\ \hline
        Парирование и блок                                                                                                           &               & *         \\ \hline
        Контр-атака                                                                                                                  &               & 1         \\ \hline
        \end{tabular}
        \end{table}

        \begin{multicols}{2}
            \textbf{Дальность оружия.}
    
            Дальность оружия обозначает количество СМ, на которое может бить оружие. 
            Разница в дальности даёт вам или вашему оппоненту Благословения или Проклятия

            \textbf{Возвышенность.}
    
            Нужно иметь позицию как минимум на 1 СМ выше чем оппонент.

            \textbf{Нападение}
    
            Если персонаж в ходе движения в пределах дальности оружия оппонента, он может инициировать 
            схватку и провести проверку Ближнего боя в рамках своей активации. При этом он получает бесплатное Благословение.

            Однако если персонаж уже был в пределах дальности оппонента (и дальность вашего оружия меньше),
            то вы не можете получить Благословение.

            \textbf{Парирование и блок.}
    
            Обороняющийся персонаж не стоит на месте, а пытается защищаться. Вы получаете штраф к своей проверке  
            Ближнего боя за каждые 5 единиц в характеристике Ближнего боя оппонента (+2 если Ближний бой оппонента равен 10).
            Заметка: вы не получаете этот штраф, если противник Оглушён или вы атакуете оппонента из вне арки видимости.

            \textbf{Контр-атака.}
    
            Если атакующий проваливает проверку ближнего боя, то защищающийся может немедленно атаковать в ответ. 
            Это не затрагивает количество действий защищающегося. 1 боец может проводить только 1 контр-атаку в ход.
            Боец может контр-атаковать, только если он в пределах дистанции противника.
            Если он стоит спиной к атакующему, он автоматически поворачивается лицом к оппоненту.
            Контр-атака даёт Проклятие.
    
        \end{multicols}
