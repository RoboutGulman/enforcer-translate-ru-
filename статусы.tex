\begin{multicols}{2}
    [
    \section{Статусы}
    ]    

    \textbf{Страх}

    В определённой ситуации ГМ может сказать игроку пройти проверку Will, чтобы сопротивлятся страху. Вместо своей хар-ки Will персонаж может использовать хар-ку Leadership 
    дружественного персонажа в пределах зоны видимости.

    К таким ситуациям можно отнести:

    \begin{itemize}
        \item Когда боец теряет HP
        \item Когда боец получает статус Горит
        \item Когда боец получает попадание от психосилы
        \item Когда дружественный боец погибает
    \end{itemize}

    Окончательное решение, когда проходить проверку на страх, выносит ГМ.

    Когда персонаж находится в статусе Страх, он может совершать только 1 действие за активацию, и ему доступны только следующие действия:

    \begin{itemize}
        \item \textbf{Сражаться} совершить стрелковую атаку против ближайшего видимого противника
        \item \textbf{Убегать} убегать к ближайшему укрытию (если уже в укрытии, убегать как можно дальше от ближайшего противника)
        \item \textbf{Замереть} совершить действие Залечь в укрытие
    \end{itemize}

    Как только персонаж получает статус Страх, он незамедлительно совершает одно из этих действий. Случайно определите.

    При последующих активациях, пройдите повторную проверку Will. При успехе, статус снимается. При провале, случайно определите, какое из 3-ёх доступных действий совершает персонаж.

    \textbf{Кровотечение}

    Персонаж с этим статусом теряет 2 HP в конце каждой своей активации (вне зависимости от оставшихся AP). Данный статус можно убрать с помощью действия Медицинская помощь.

    \textbf{Отравление}

    Персонаж с этим статусом теряет 1 HP в конце каждой своей активации (вне зависимости от оставшихся AP). При этом он получает -2 см к своей скорости. В конце каждой своей активации, до получения урона, боец должен пройти тест стойкости. При успехе, он снимает с себя состояние Отравлен.

    \textbf{Горит}

    Горение длится 3 хода, и наносит 2 урона каждый ход. Боец с этим свойством ополовинивает свою скорость, и каждый ход он должен проходить тест Will, чтобы сопротивляться страху.
    Данное состояние может быть снято Медицинской помощью или починкой брони, а также водой и другими подходящими средствами.

    \textbf{Оглушён}

    Оглушённый боец теряет одно действие в ходе текущей или следующей активации. Данное состояние может накапливаться (например, дважды оглушённый боец 
    пропустит активацию целиком).

    \textbf{Дезориентирован}

    Статус длится 2 хода и пока боец находится в этом статусе, он получает проклятие к BS, WS и проверкам наблюдательности.
    В дополнении, боец не может совершать ни одно из следующих действий: прицеливание, предупреждающий огонь, подавляющий огонь, пси-сила, медицинская помощь, починка брони.

    \textbf{Прижат огнём (Pinned)}

    Причиной данного состояния является подавляющий огонь. Персонаж с этим статусом получает проклятие к BS и не может совершать 
    действие прицеливание. В дополнение, движение или стрельба подавленного огнём персонажа вызывает незамедлительную стрелковую 
    атаку от того, кто производит огонь на подавление.

    \textbf{Тяжелоранен}
    
    Состояние накладывается, когда HP бойца падают до 0. Боец считается имеющим статус Оглушён (т.е. он пропускает все свои действия), пока с него не будет снят статус Тяжелоранен.

    \textbf{При смерти}

    Боец считается имеющим статус Оглушён (т.е. он пропускает все свои действия), пока с него не будет снят статус При смерти. 
\end{multicols}