\begin{multicols}{2}
    [
    \section{Урон}
    ]    

    \textbf{Получение урона}

    Если не учитывать спец. правила, сначала весь урон вычитается из АР (очков брони) бойца, и только затем из его HP. 

    Если боец упал до 0 здоровья, он получает статус тяжелоранен. Если здоровье меньше нуля 
    и при этом число больше чем половина макс. здоровья (-3 для 6 ХП), он получает статус Умирает.

    Если статус Умирает длится 3 хода, или здоровье бойца меньше нуля и число превысило максимальное количество здоровья, боец мёртв.
    
    Здоровье и броню можно восстановить с помощью соответствующих действий, но они не могут превышать максимального значения.

    \textbf{Свойства оружия}

    Множество оружия имеет свойства, которые накладывают вторичные состояния на бойцов. Если в бойца успешно попало оружие с таким свойством, пройдите 
    тест Стойкости. Статус накладывается на бойца только при провале этого теста.

    \textbf{Урон от падения}

    Боец может упасть на дистанцию вплоть до половины своей хар-ки Ловкости не получая урона (например, 5см для хар-ки 10). Если высота
    превышает допустимую, боец получает 1 урон за каждый см. При этом он должен пройти тест на стойкость чтобы не получить статус Оглушён.
\end{multicols}