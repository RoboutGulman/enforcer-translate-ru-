\begin{multicols}{2}
    [
    \section{Урон}
    ]    

    \textbf{Получение урона}

    Если не учитывать спец. правила, сначала весь урон вычитается из АР (очков брони) бойца, и только затем из его HP. 

    Если боец упал до 0 здоровья, он получает статус тяжелоранен. Если здоровье меньше нуля 
    и при этом число больше чем половина макс. здоровья (-3 для 6 ХП), он получает статус Умирает.

    Если статус Умирает длится 3 хода, или здоровье бойца меньше нуля и число превысило максимальное количество здоровья, боец мёртв.
    
    Здоровье и броню можно восстановить с помощью соответствующих действий, но они не могут превышать максимального значения.

    \textbf{Свойства оружия}

    Множество оружия имеет свойства, которые накладывают вторичные состояния на бойцов. Если в бойца успешно попало оружие с таким свойством, пройдите 
    тест Стойкости. Статус накладывается на бойца только при провале этого теста.

    \textbf{Урон от падения}

    Боец может упасть на дистанцию вплоть до половины своей хар-ки Ловкости не получая урона (например, 5см для хар-ки 10). Если высота
    превышает допустимую, боец получает 1 урон за каждый см. При этом он должен пройти тест на стойкость чтобы не получить статус Оглушён.

    \textbf{Травмы}

    Персонаж, здоровье которого в ходе сценария упало до половины макс. HP или ниже, может получить травму. Персонажи не могут участвовать в сцеариях, пока они ранены. 
    
    После окончания сценария, персонажи, который в любой момент сценария имели половину от макс. HP или меньше, должны пройти тест на Стойкость. При провале, они получают лёгкую травму.
    
    Персонажи, которые в любой момент сценария имели статус Тяжелоранен, должны пройти тест на Стойкость. При провале, они получают среднюю травму.

    Персонажи, которые в любой момент сцеария имели статус При смерти, должны пройти тест на Стойкость. При провале, они получают критическую травму.

    При получении травмы персонажи должны пропустить определённый срок, прежде чем вновь смогут участвовать в сценариях.

    Лёгкая травма - d6 дней.

    Средняя травма - 2d6 дней.

    Критическая травма - 3d6 дней.

    Если персонаж получил критическую травму, выберите одну случайную хар-ку и понизьте её значение на 1: HL, WS, BS, Dx, T

    Если во время проверки на стойкость бросок был 1 или 20, Will бойца понижается на 1.

    \textbf{Повреждённая броня}

    Если на конец сценария, AP бойца понизились до 0, его броня больше не может быть использована. В конце сценария, любой персонаж в команде может 
    пройти тест на интллект, чтобы починить броню до след. сценария. Эта проверка получает +1 за каждое AP, которое броня имела при покупке. За один раз перрсонаж может попытаться 
    починить только 1 элемент брони.

    При успехе броска, броня восстановлена. При провале, она непригодна для использования на следующие 2d6 дней.

\end{multicols}